\documentclass[a4paper,12pt]{article}

\usepackage{cmap}
\usepackage[T2A]{fontenc}
\usepackage[utf8]{inputenc}
\usepackage[english,russian]{babel}
\usepackage{longtable}

%\author{Семенников Кирилл, Чистяков Александр}
\newcommand{\specialcell}[2][l]{%
  \begin{tabular}[#1]{@{}l@{}}#2\end{tabular}}
%\title{Устав проекта \\"Организация фестиваля Пива и Молока"}
%----------------------------------------------------------------------------------------
%   TITLE PAGE
%----------------------------------------------------------------------------------------

\newcommand*{\titleGM}{\begingroup % Create the command for including the title page in the document
  \hbox{ % Horizontal box
    \hspace*{0.1\textwidth} % Whitespace to the left of the title page
    \rule{1pt}{\textheight} % Vertical line
    \hspace*{0.05\textwidth} % Whitespace between the vertical line and title page text
    \parbox[b]{0.85\textwidth}{ % Paragraph box which restricts text to less than the width of the page

      {\noindent\Huge\bfseries Устав проекта\\[0.5\baselineskip] ''Организация\\фестиваля Пива и Молока''}\\[2\baselineskip] % Title
      {\large \textit{в Санкт-Петербуре}}\\[4\baselineskip] % Tagline or further description
      {\Large \textsc{Семенников Кирилл,\\Чистяков Александр}} % Author name

      \vspace{0.5\textheight} % Whitespace between the title block and the publisher
      {\noindent 2013}\\[\baselineskip] % Publisher and logo
    }}
\endgroup}
\begin{document}
%\maketitle
\thispagestyle{empty} % Removes page numbers
\titleGM % This command includes the title page
\newpage
\section{Общая информация}
\begin{center}
\begin{tabular}{|l|l|}
  \hline
  Инициатор & ООО ''Вектор''\\
  \hline
  Заказчик & \specialcell{Комитет по здравоохранению\\Администрации Санкт-Петербурга}\\
  \hline
  Ожидаемые сроки реализации & \\
  \hline
  Начало: 21/01/2014 & Конец: 04/04/2014\\
  \hline
\end{tabular}
\end{center}
\section{Цели проекта}
\begin{enumerate}
  \item{Оздоровление населения города Санкт-Петербурга.}
  \item{Формирование имиджа города Санкт-Петербурга как культурной столицы России.}
  \item{Эффективное перераспределение бюджетных средств.}
\end{enumerate}
\section{Обоснование целесообразности проекта}
Целесообразность реализации проекта обусловлена следующими факторами:
\begin{itemize}
  \item Снижение показателя среднего уровня счастья жителей Санкт-Петербурга, что ведет к сокращению продолжительности жизни населения.
  \item Необходимость поддержки внешнего облика Санкт-Петербурга как культурной столицы России.
  \item Результаты исследования британских ученых о чудодейственных последствиях  одновременного употребления Молока и Пива.
  \item Снижение среднего показателя употребления алкоголя среди молодежи, что ведет к пониманию реальных проблем Государства и росту уровня социальной напряженности.
  %\item Рост популярности проектов Фонда борьбы с коррупцией, что ведет к усложнению проведения мошеннических схем.
\end{itemize}
Таким образом, необходимо провести фестиваль Пива и Молока в Санкт-Петербурге, который позволит:
\begin{itemize}
  \item Повысить индекс счастья населения Санкт-Петербурга.
  \item Повысить среднюю продолжительность жизни населения Санкт-Петербурга.
  \item Улучшить имидж Санкт-Петербурга и привлечь туристов из развитых стран.
  \item Снизить уровень оппозиционных настроений среди населения Санкт-Петербурга.
  %\item Обоготить всех участников проекта посредством оказания услуг, несоответствующих выделенным бюджетным средствам.
\end{itemize}
\section{Основные этапы проекта}
Запуск проекта должен быть произведен до 21/01/2014
\begin{enumerate}
  \item{Планирование фестиваля --- 18 дней}
  \item{Подготовка к проведению фестиваля --- 31 день}
  \item{Проведение фестиваля --- 1 день}
  \item{After party --- 3 дня}
\end{enumerate}
\begin{center}
\begin{longtable}{|p{4cm}|p{11cm}|}
  \hline
  \multicolumn{1}{|c|}{\textbf{Этап}} & \multicolumn{1}{|c|}{\textbf{Задачи}}\\
  \hline
  \endhead
  Планирование фестиваля & 
  \begin{enumerate}
    \item Нахождение и резервирование площадки для проведения фестиваля
    \item Нахождение поставщиков ресурсов: Пива различных сортов, Молока от различных коров, медикаментов для для пищеварения 
    \item Разработка плана развлекательной программы 
    \item Разработка плана рекламной кампании
    \item Разработка плана обеспечения безопасности
  \end{enumerate} \\
  \hline
  Подготовка к проведению фестиваля & 
  \begin{enumerate}
    \item Оборудование площадки 
      \begin{enumerate}
        \item Оборудование вывески с лампочкой
        \item Установка пункта обмена Пива и Молока на деньги 
        \item Организация системы управления отходами:
          \begin{itemize}
            \item Установка биотуалета
            \item Высаживание кустов
            \item Тестирование системы управления отходами
          \end{itemize}
      \end{enumerate}
    \item Реализация рекламной кампании
  \end{enumerate} \\
  \hline
  Проведение фестиваля &
  \begin{enumerate}
    \item Продажа Пива и Молока
    \item Проведение развлекательной программы аниматором
    \item Обеспечение безопасности во время фестиваля
    \item Оказание медицинской помощи пострадавшим
  \end{enumerate} \\
  \hline
  After party & 
  \begin{enumerate}
    \item Уборка территории
    \item Уборка территории
    \item Уборка территории
    \item Распределение нереализованных ресурсов
      \begin{itemize}
        \item Перемаркировка даты окончания срока годности на тарах с молоком с целью дальнейшей перепродажи
        \item Сбыт оставшегося молока
        \item Разбавление оставшегося пива водой с целью проведения второго фестиваля
      \end{itemize}
  \end{enumerate}\\
  \hline
\end{longtable}
\end{center}
\section{Ресурсы проекта}
\subsection{Трудовые ресурсы}
\begin{tabular}{|l|c|l|l|p{3cm}|}
  \hline
  \multicolumn{1}{|c|}{\textbf{Ресурс}} & \multicolumn{1}{|c|}{\textbf{Кол.}} & \multicolumn{1}{|c|}{\textbf{Вид}} & \multicolumn{1}{|c|}{\textbf{Вид оплаты}} & \multicolumn{1}{|c|}{\textbf{Ставка}}\\
  \hline
  \specialcell{Деловой\\представитель} & 1 & \specialcell{Штатный\\сотрудник} & Тарифная ставка & \specialcell{2800 руб./день\\(2 стип./день)}\\
  \hline
  Аниматор & 1 & \specialcell{Штатный\\сотрудник} & Тарифная ставка & \specialcell{1400 руб./день\\(1 стип./день)}\\
  \hline
  Полицейский & 2 & Приглашенный & \specialcell{Разово\\по окончании работы} & \specialcell{67200 руб.\\(стип. за 4 года)}\\
  \hline
  \specialcell{Специалист\\по рекламе} & 1 & \specialcell{Штатный\\сотрудник} & Тарифная ставка & \specialcell{2800 руб./день\\(2 стип./день)}\\
  \hline
  Специалист АХО & 1 & \specialcell{Штатный\\сотрудник} & Тарифная ставка & \specialcell{700 руб./день\\(0.5 стип./день)}\\
  \hline
  \specialcell{Инженер\\по тестированию} & 1 & \specialcell{Штатный\\сотрудник} & Тарифная ставка & \specialcell{462 руб./день\\(0.33 стип./день)}\\
  \hline
  Продавец & 2 &  \specialcell{Штатный\\сотрудник} & Тарифная ставка & \specialcell{700 руб./день\\(0,5 стип./день)}\\
  \hline
  Врач & 1 & Приглашенный & \specialcell{Разово\\по окончании работы} & \specialcell{16800 руб.\\(стип. за год)}\\
  \hline
\end{tabular}
\section{Риски проекта}
\begin{itemize}
  \item Получение информации о проекте Фондом борьбы с коррупцией.
  \item Увеличение сроков согласования документов для резервирования площадки проведения фестиваля.
  \item Неквалифицированное выполнение сотрудниками полиции своих обязанностей.
\end{itemize}
\section{Ограничения}
\begin{center}
\begin{tabular}{|p{4cm}|p{11cm}|}
  \hline
  \specialcell{Время исполнения\\проекта} & 73 дня\\
  \hline
  Затраты по проекту & 
  \begin{itemize}
    \item Стоимость материальных ресурсов --- не более 500000 руб.
    \item ФОТ соттрудников компании --- 222600 руб.
    \item Премия сотрудникам --- 100000 руб.
  \end{itemize}\\
  \hline
  Организационные &
  \begin{itemize}
    \item Место проведение фестиваля и количество человек должно быть согласовано с властями города.
    \item Необходимо предварительное получение алкогольной лицензии и молочной лицензии.
  \end{itemize}\\
  \hline
  Время команды проекта & Все участники проекта выделяют 146\% своего рабочего времени.\\
  \hline
\end{tabular}
\end{center}
%\subsection{Ограничения по ресурсам}
\end{document}
