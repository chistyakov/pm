\documentclass[a4paper,12pt]{article}

\usepackage{cmap}
\usepackage[T2A]{fontenc}
\usepackage[utf8]{inputenc}
\usepackage[english,russian]{babel}

%\usepackage[inline]{enumitem}

\author{Семенников Кирилл, Чистяков Александр}
\title{Устав проекта "Организация фестиваля Пива и Молока"}
\begin{document}
\maketitle
\section{Цели проекта}
\begin{enumerate}
  \item{Оздоровление населения}
  \item{Формирование имиджа города Санкт-Петербурга как культурной столицы}
  \item{Отмывание бюджетных средств}
\end{enumerate}
\section{Требования к проекту}
\section{Основные этапы проекта}
\begin{enumerate}
  \item{Планирование фестиваля}
  \item{Подготовка }
  \item{After party}
\end{enumerate}
\begin{tabular}{|c|p{10cm}|}
  \hline
  \textbf{Этап} & \multicolumn{1}{|c|}{\textbf{Задачи}}\\
  \hline
  Планирование фестиваля & 
  \begin{enumerate}
    \item Нахождение и резервирование площадки для проведения фестиваля
    \item Нахождение поставщиков ресурсов: пива различных сортов, молока от различных коров, медикаментов для для пищеварения 
    \item Разработка развлекательной программы 
    \item Разработка рекламной кампании
  \end{enumerate} \\
  \hline
  Подготовка территории проведения фестиваля& 
  \begin{enumerate}
    \item Установка  
    \\
  \hline
  After party & 
  \begin{enumerate}
    \item Уборка территории
    \item Уборка территории
    \item Уборка территории
    \item Распределение нереализованных ресурсов
      \begin{itemize}
        \item Перемаркировка даты окончания срока годности на тарах с молоком с целью дальнейшей перепродажи
        \item Разбавление оставшегося пива водой с целью проведения второго фестиваля
      \end{itemize}
  \end{enumerate}\\
  \hline
\end{tabular}
\section{Риски}
\section{Ограничения}
\subsection{Ограничения по времени}
\subsection{Ограничения по финансам}
\subsection{Ограничения по ресурсам}
\end{document}
